\documentclass[12pt]{article}

\title{Research proposal: idiolects in word prediction}
\author{Wessel Stoop}

\usepackage{graphicx}
\usepackage{float}
\usepackage{apacite}

\renewcommand{\familydefault}{\sfdefault}

\begin{document}

\section{Project title}
Idiolects in word prediction

\section{Summary}
[[]]

\section{Candidate}
Wessel Stoop

\section{Description of the proposed research}

This research is an attempt to link two academic fields: the field of \emph{word completion} and the field \emph{idiolects}. They will be introduced in section \ref{wordprediction} and \ref{idiolects} respectively. Section \ref{link} will show what the advantages of linking these two concepts are.

\subsection{Word prediction} \label{wordprediction}

\subsubsection{Introduction}

Word prediction is used in many applications we use daily: Google completes our search queries, our webbrowser finishes the web addresses we type in, our email client finishes email addresses, Excel completes annotations based on earlier annotations, etc. It reduces the number of keystrokes we have to do, and thus saves us time and effort. However, when we type full sentences, whether they are part of an email, a scientific article, a tweet, or a research proposal, we often still key in every single character. 

Why don't we use word prediction here too? While, as far as I am aware, no quantitative research has been done to answer this question, my own experience with word prediction is that its prediction accuracy generally is too low to speed up typing. Keying in every character is often faster than checking and accepting the predictions. Once typing speed decreases, however, word prediction \emph{is} used. With the rise of smartphones,  on which one generally cannot type as fast as on a normal keyboard, word prediction has become more widely known - although it is unclear what proportion of the smartphone users also use some sort of word prediction application. Another example are  people with speech and motor disabilities, like cerebral palsy or hemiplexia. By using a device equiped with word prediction technology, they can increase their communication rate considerably \cite{Garay-Vitoria}.

In the academic literature, several attempts have been made to improve word prediction technology, so that it can increase the communication rate for smartphone users, people with physical disabilities and typing in general. 

% WOORDENBOEK, CS, Antal & Suzan?

The goal of this research is to improve the quality of word prediction. We will build a word prediction system that incorperates the latest insights from the field, and ...

% autocompletion
%N. Garay-Vitoria and J. Abascal. 2006. Text prediction systems: a survey. Universal Access in the Information Society, 4(3):188–203.


\subsection{Idiolects} \label{idiolects}


\subsection{Idiolects in word prediction} \label{link}


\section{Summary in keywords}

\end{document}