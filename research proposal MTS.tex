\documentclass[12pt]{article}

\title{Research proposal: idiolects in word prediction}
\author{Wessel Stoop}

\usepackage{graphicx}
\usepackage{float}
\usepackage{apacite}

\renewcommand{\familydefault}{\sfdefault}

\begin{document}

\section{Project title}
Idiolects in word prediction

\section{Summary}
[[]]

\section{Candidate}
Wessel Stoop

\section{Description of the proposed research}

This research is an attempt to link two academic fields: the field of \emph{word completion} and the field \emph{idiolects}. They will be introduced in section \ref{wordprediction} and \ref{idiolects} respectively. Section \ref{link} will show what the advantages of linking these two concepts are.

\subsection{Word prediction} \label{wordprediction}

\subsubsection{Introduction}

Word prediction is used in many applications we use daily: Google completes our search queries, our webbrowser finishes the web addresses we type in, our email client finishes email addresses, Excel completes annotations based on earlier annotations, etc. It reduces the number of keystrokes we have to do, and thus saves us time and effort. However, when we type full sentences, whether they are part of an email, a scientific article, a tweet, or a research proposal, we often still key in every single character. This can be time consuming for regular users, but highly frustrating for people with physical disabilities; [[example]]. Word prediction is not used for sentences because it is generally considered 'not good enough'; having to reject wrong predictions all the time apparently is more frustrating than having to key in every character. With the rise of smartphones during the last years word prediction system have become more widely known (because it takes more time to type on a phone, which makes autocompletion more attractive), although it is unclear what portion of the smartphone users also use some sort of autocompletion application.

The goal of this research is to improve the quality of word prediction. We will build a word prediction system that incorperates the latest insights from the field, and ...

% autocompletion
X
%N. Garay-Vitoria and J. Abascal. 2006. Text prediction systems: a survey. Universal Access in the Information Society, 4(3):188–203.


\subsection{Idiolects} \label{idiolects}


\subsection{Idiolects in word prediction} \label{link}


\section{Summary in keywords}

\end{document}