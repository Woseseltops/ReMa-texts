\documentclass[12pt]{article}
\title{Research training portfolio}
\author{Wessel Stoop}

\begin{document}
\maketitle

\section{Introduction}

\section{What I have done}

\subsection{Project structure}
\subsection{Problems encountered}
\subsection{Results}

\section{What I have learned}
Although I did have some experience with both programming and language technology, and even with some software co-developed by my supervisor, this internship turned out to be a adventures full of novelty: no single semester in my Radboud University career had such a high skills-acquired/day-ratio as this one.

\subsection{Working with Linux and SSH}
The novelty already started with something as non-trivial as the choice of operating system. As a Microsoft Windows user, I was surprised to find out all machine learning software I needed was developed and best used on Linux (see the next section). I only had a few hours of experience with Ubuntu, so I had to invest several hours into installing and getting more familiar with it. Most time went into learning the terminal commands - its GUI turned out to be almost identical to Windows'. Windows of course has a command prompt as well, but it is rarely used. Ubuntu's commands on the other hand are essential for having full control. The first hours of the internship were therefore spent figuring out basic actions like \emph{ls} (showing folders and files in a directory) and \emph{mv} (renaming and moving files). This knowledge later turned out to be incredibly handy when I needed to control external servers over an SSH connection, as the Linux machines can be controller with commands only. Here again, it took some time to figure out how to do this properly. 

% Maar.... nu kan ik het heel goed?
% Voordelen Linux?

\subsection{Working with machine learning software}
Fowlt uses the machine learning software TiMBL, TiMBLserver and WOPR.

% Namen ontwikkelaars.
% With the help of my supervisors

\subsection{Working with GitHub}
Working with multiple people on the same software can cause quite a lot of practical problems. Version control system Git solves these problems by keeping track of which version everyone of working on, providing tools to easily merge the work when done, and to remember older versions of the project, among other things. Although I had used Git for the latter function in personal hobby projects, I'd never worked together with other people on programming tasks. During this internship I learned (1) this can be done easily with GitHub and (2) how to handle GitHub. GItHub is a website that offers open source project a free, central and always reachable place for a Git repository, and also provides tools to look into this repo and its history online.

\subsection{Working with CLAM, setting up a NLP webservice}

\subsection{General}

\section{Research paper}

\subsection{Introduction}

\subsection{Method}

\subsection{Results}

\subsection{Conclusion}

\section{General onclusion}

\end{document}