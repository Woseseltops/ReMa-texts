\documentclass[12pt]{article}

\title{Modelling periphrastic \emph{doen} in Dutch}
\author{Wessel Stoop, s0808709}

\usepackage{covington}
\usepackage{graphicx}
\usepackage{natbib}
\usepackage{gb4e}

\renewcommand{\familydefault}{\sfdefault}
\let\eachwordtwo=\sf

\begin{document}
\maketitle

\section{Introduction}
In the Dutch main clause, there are two verb positions: the \emph{verb second position} and the \emph{verb final} position. The verb second position is used for the inflected verb, which means that it is always filled; the verb final position is only used when there is more than one verb. This is exemplified in example 1 and 2.

\begin{exe}
\ex \gll Ik \textbf{eet} koekjes\\
 'I \textbf{ate} cookies'\\

\ex \gll Ik \textbf{heb} koekjes \textbf{gegeten}\\
I \textbf{have} cookies \textbf{eaten}\\
\trans 'I have eaten cookies'
\end{exe}

In standard Dutch, no such thing as do-support is known. For example, subject-auxiliary inversion is possible with all verbs to form questions.

\begin{exe}
\ex \gll Eet ik koekjes? \\
Eat I cookies? \\
\trans 'Did I eat cookies?
\end{exe}

Interestingly, however, in some non-standard, informal varieties of Dutch, something very similar to do-support can be found. Speakers replace the inflected verb in verb second position with an inflected form of \emph{doen} 'do', and place the meaningful verb in verb final position. The \emph{doen}-variant of example 1 would be:

\begin{exe}
\ex \gll Ik doe koekjes eten \\
I do cookies eat\\
\trans 'I eat cookies'
\end{exe}

\citet{g83} notes this construction with \emph{doen}, often called \emph{periphrastic doen}, is not popular with language users - or at least the ones aware of the construction: 'language users consider [it] unacceptable, and such sentences are mostly associated with childish language. [It is a] frequent phenomenon in spoken language, contested on the basis of non-linguistic arguments or prescriptive norms' (\citealp[p. 58]{g83}, translation WS). On \emph{taaladvies.net}, a Dutch website giving language-related advice, the use of periphrastic \emph{doen} is strongly discouraged because it 'does not add anything substantial' (translation WS).\\\indent
This dislike seems to be shared by some linguists too. \citet[p. 121]{d94} for example says: 'experienced language users do not have trouble with larger formal complexity; they don't need the ease of paraphrasing. It is because of this [periphrastic \emph{doen}] is devious' (translation WS), and \citet[p. 153]{n62} claims the construction is mainly limited to children and women.\\\indent
Interestingly, the construction has also been claimed to be part of dialectal language use: it would be part of the dialects of Groningen \citep{tl53}, Twente \citep{n62} and Heerlen \citep{c94}, dialects that have in common that they are all spoken in the east of the Netherlands.\\\indent
In this paper, I will show that periphrastic \emph{doen} is not limited to dialectal language or the language of children or women,  but seems to be part of (the eastern variant of) Dutch in general. Furthermore, I will try to explain what triggers its use. For this, I will use both a regression model and a model based on memory based learning.\\\indent
The remainder of this paper consists of 4 sections. In section \ref{data}, I will elaborate on the collection and the nature of the dataset used. Section \ref{regression} consists of an overview of factors that could possibly have an influence on whether periphrastic \emph{doen} is used or not, and to what extent these factors indeed can explain the data. In section \ref{mbl}, I will explain the workings of Memory Based Learning software TiMBL, and discuss its results on predicting in which cases periphrastic \emph{doen} is used. The paper will be concluded in section \ref{conc}.

\section{The dataset} \label{data}

The dataset was collected from the Corpus Gesproken Dutch ('Corpus Spoken Dutch'), henceforth CGN \citep{cgn}. The CGN contains different types of speech, including spontaneous face-to-face telephone conversations, interviews, debates, radio shows and read-aloud books. Part of the corpus was collected around Nijmegen, a city in the east of the Netherlands, making the CGN suitable for this research.\\\indent
To obtain all examples of periphrastic \emph{doen}, the following query was used:

\begin{exe}
\ex lemma DOEN as finite verb + infinitive within 10 words
\end{exe}

This query is not perfect, both in terms of recall and precision: because of the limitation of 10 words, sentences with more material between the two verbs are missed, and this query returns a lot of examples which are not related to periphrastic \emph{doen}. I present two examples of such false alarms:

\begin{exe}
\ex \gll wat doe jij hier eigenlijk als ik vragen mag? \\
what do you here actually if I ask may?\\
\trans 'What are you doing here, if I may ask?'

\ex \gll en die jeugdboeken dat doe 'k {nog wel} …PAUSE… zo in een ruk uitlezen \\
and that {children's books} that do I still ...Pause... so in one pull finish\\
\trans 'For children's books I still do it, finishing them in one sitting'
\end{exe}

Although these examples indeed contain the lemma \emph{doen} as finite verb followed by an infinitive within ten positions, they are not examples of periphrastic \emph{doen} because the finite \emph{doen} here does not replace the finite verb; instead, the two verbs are part of different clauses.\\\indent
The resulting set of sentences was cleaned manually. A total of 66 usable examples of periphrastic \emph{doen} remained. Three examples:

\begin{exe} 
\ex \gll doe jij morgen de hond uitlaten of moet ik het doen? \\ 
do you tomorrow the dog walk or must I it do?\\ \label{hond}
\trans 'Do you walk the dog tomorrow, or should I do it?'

\ex \gll dat doe ik {dan wel} kijken als ik dan toch voor Alex uh even kom kijken \\
that do i then look when I then anyway for Alex er momentarily come look\\ \label{Alex}
\trans 'I'll look at that when I come look for Alex'
\end{exe}

To these 66 examples, 100 randomly selected 'control' sentences were added. These are sentences with a finite verb but without periphrastic \emph{doen}, which is almost every Dutch sentence. Picking a (sub)corpus to select these control sentences from turned out to be a rather difficult task. \\\indent
For example, if 100 sentences selected from all Dutch corpora available would have been used, and then annotated for register, we would probably see a significant effect for register: the \emph{doen} examples are all spoken, informal language, while this would not true for all control sentences. We would thus find that there is no variation in formal language. However, if we then wanted to use this same dataset to investigate other effects, these formal sentences might obscure the view; for example, while verb type might be a strong predictor for the use of periphrastic \emph{doen} in informal language, it is not at all in formal language (as it never occurs there, regardless of verb type). Including formal sentences thus decreases the size of the effect. In other words, because the variation does not occur in formal language, it seems to make sense to exclude formal language, so we can zoom in on the place where the variation \emph{does} occur. \\\indent 
On the other hand, if it turned out that periphrastic \emph{doen} is limited to the verb \emph{zijn} 'be', the researcher could decide to also only pick control sentences with \emph{zijn}. While doing this allows him or her to discover very specific effects about the variation, it is no longer possible to discover the general effect that \emph{doen} always co-occurs with \emph{zijn}, while this might be a very important predictor for periphrastic \emph{doen} and part of the explanation why it is used. In other words, when being too picky in selecting the control sentences, one risks throwing away the baby with the bathwater.\\\indent
Because the literature about periphrastic \emph{doen} is so limited, I decided to focus on more general effects. There does not seem to be any consensus about possible predictors besides that it is limited to informal language, so that is the only place the control sentences were limited: all were taken from the \emph{face-to-face conversations} or \emph{telephone conversations} from the CGN, because that is also the place where all \emph{doen} sentences came from (with the exception of one example from sports commentaries).

% Envelope of variation
% Misschien iets doen met interacties?

\section{The regression model} \label{regression}

\subsection{Factors included}

In this section, I will elaborate on the factors included in the regression model, and use chi-square tests to see whether the \emph{doen} sentences behave differently from the control sentences with the regard to the concept investigated. In section \ref{regressionresults} the predictive power of all these factors combined will be discussed.

\subsubsection{Phrasal verbs}
In Dutch, several verbs consists of multiple parts. \emph{Uitschelden} 'abuse, insult', for example, consists of \emph{uit} 'out' and \emph{schelden} 'scold'. It should be noted, however, that these verbs are relatively infrequent compared to normal verbs; in the frequency list based on the materials of the website \emph{http://www.opensubtitles.org}, the first phrasal verb encountered is \emph{voorstellen} 'imagine, introduce' at position 876, and others are even further down the list (for example, \emph{paardrijden} 'horse riding': 11341, \emph{opleggen} 'impose, lay on': 13577, \emph{uitschelden} 'abuse, insult': 24604).\\\indent
When phrasal verbs are used as a finite verb in the main clause, only the verbal part appears on the second position. The rest of the verb appears in the sentence-final position.

\begin{exe}
\ex \gll Ik \textbf{stel} haar aan je \textbf{voor}\\
I introduce her to you\\
'I'll introduce her to you'
\end{exe}

\citet[pp. 156]{n62} suggests periphrastic \emph{doen} is a strategy to avoid the extra processing load this long-distance relationship causes, both for the speaker and the hearer. If that is true, we would expect to find much more phrasal verbs as the main verb in the \emph{doen} sentences than in the control sentences. This turns out to be the case:

\begin{table}[h]
\begin{tabular}{|l||ll|ll|}
\hline
&Phrasal verbs&&Non-phrasal verbs&\\
\hline
periphrastic \emph{doen} sentences	&23 	& \emph{34.85\%}	&43		 &\emph{65.15\%}\\
control sentences					&0		& \emph{0\%}		&100	 &\emph{100\%}\\
\hline
\end{tabular}
\end{table}

Because of their infrequency, phrasal verb do not even occur in the control sample, while one-third of the main verbs used with periphrastic \emph{doen} is a phrasal verb. A chi-square test showed this difference is indeed significant ($\chi^2(1, N=166) = 40.45, p > .001$).

\subsubsection{The given before new-principle}

Although there is a lot of disagreement over the exact details and the terminology, there is a wide consensus among linguists that languages often use the \emph{given before new}-principle (ref). This principle entails that speakers try to put information already known by the addressee in the beginning of the sentence, while the part of the sentence that contains the new information is put at the end as much as possible. Or, to put it differently, speakers first introduce their topic, and then say what they want to say about this topic.\\\indent
%ref
The Dutch V2-rule, however, limits the possibilities to do this drastically. Unless there is an auxiliary present for other reasons, it is not possible to put the main verb at the end of the sentence in main clauses. periphrastic \emph{doen} could be a perfect tool to solve this problem, as it forces the main verb to go to the end of the sentence. If \emph{doen} is indeed used this way, we would expect that the main verb in \emph{doen} sentences indeed is new information. To investigate this, the conversations preceding all sentences were listened to. If the concept the main verb conveys was already mentioned in some form, the main verb was annotated as old information, whereas it was annotated as new information if it was not mentioned somehow before.

\begin{table}[h]
\begin{tabular}{|l||ll|ll|}
\hline
&Old information&&New information&\\
\hline
periphrastic \emph{doen} sentences	&15 	& \emph{22.73\%}	&51		 &\emph{77.27\%}\\
control sentences					&24		& \emph{24\%}		&76		 &\emph{76\%}\\
\hline
\end{tabular}
\end{table}

As can be seen from the table, there does not seem to be a clear difference between the \emph{doen} sentences and the control sentences; in both cases the main verb was new information most of the times, regardless of whether \emph{doen} was used. A chi-square test showed this difference was not significant ($\chi^2(1, N=166) = 0.036, p = .85$).

\subsubsection{Habitual aspect}

\citet{c94} discusses periphrastic \emph{doen} as part of the Heerlen dialect, and claims it has a semantic function rather than being a syntactic strategy; periphrastic \emph{doen} would be used to express habitual aspect. Although such a categorical explanation does not seem to work for all examples used in the current dataset, (a habitual interpretation for sentences like \ref{hond} and \ref{Alex} would be very unlikely, for example), it might also play a role here. To investigate this, for every sentence it was annotated whether a habitual interpretation was possible. It should be added that the possibility of such an interpretation does not necessarily mean the speaker also intended it.

\begin{table}[h] 
\begin{tabular}{|l||ll|ll|} 
\hline
&Possibly habitual&&Habitual impossible&\\
\hline
periphrastic \emph{doen} sentences	&22 	& \emph{33.33\%}	&44		 &\emph{66.67\%}\\
control sentences					&19		& \emph{19\%}		&81		 &\emph{81\%}\\
\hline
\end{tabular}
\end{table}

The data show a small difference between periphrastic \emph{doen} and the control sentences; a possible habitual doen interpretation is a little more likely when periphrastic \emph{doen} is present. A chi-square test showed this difference was significant ($\chi^2(1, N=166) = 4.39, p = .036$).

\subsubsection{Variety}

As discussed in the introduction, the literature suggests periphrastic \emph{doen} might be a phenomenon limited to eastern Dutch. Whereas the longitude of the speaker's origin cannot be checked easily for the CGN data, the latitude can: the data are annotated for being either Netherlandic or Belgian Dutch. One third of the CGN is recorded in Belgium.

\begin{table}[h] 
\begin{tabular}{|l||ll|ll|} 
\hline
&Netherlandic&&Belgian&\\
\hline
periphrastic \emph{doen} sentences	&63 	& \emph{95.45\%}	&3		 &\emph{4.55\%}\\
control sentences					&67		& \emph{67\%}		&33		 &\emph{33\%}\\
\hline
\end{tabular}
\end{table}

These data reveal periphrastic \emph{doen} seems to be mostly limited to Netherlandic Dutch. A chi-square test showed this difference was significant ($\chi^2(1, N=166) = 18.954, p > .001$).

\subsubsection{Person}

An observation done while annotating for the previously described factors is that periphrastic \emph{doen} is used in first and second person, but never in the third. While this can be linked to various more abstract explanations, as will be attempted in the next two sections, it was first investigated whether this really was the case, and whether this is expected on the basis of the control sentences.

\begin{table}[h]
\begin{tabular}{|l||ll|ll|ll|}
\hline
&First&&Second&&Third&\\
\hline
periphrastic \emph{doen} sentences	&26 	& \emph{39.39\%}	&40		&\emph{60.61\%} 	& 0 &\emph{0\%}\\
control sentences					&55		& \emph{55\%}		&21	 	&\emph{21\%} 		& 24 &\emph{24\%}\\
\hline
\end{tabular}
\end{table}

Here we see indeed that the third person is absent from the \emph{doen} sentences, which suggests that periphrastic \emph{doen} in third person is infrequent, and possibly even ungrammatical. It should be added that a counter-example to that last suggestion was found on \emph{taaladvies.net}, the language advice website discussed earlier, although it unclear to what extent this example is artificial:

\begin{exe}
\ex \gll Zij doen daar veel kaarten\\
They do there much play.cardgames\\
'They play cardgames a lot there'
\end{exe}

Another thing that strikes the eye is that second person is a lot more frequent in the \emph{doen} sentences than in the control sentences. This is related to the fact that the number of imperatives is exceptionally high among the \emph{doen} sentences. This will be investigated further in the next section.\\\indent
A possible interpretation of periphrastic \emph{doen}'s preference for the first and second person is that \emph{doen} is somehow limited to interactional uses; the lack of periphrastic \emph{doen} in third person could then be explained by the fact that the third person cannot be used for interaction. In other words, it is possible that periphrastic \emph{doen} functions as something like an 'interaction marker'.\\\indent
A chi-square test showed the difference between the \emph{doen} sentences and the control sentences was significant ($\chi^2(2, N=166) = 34.8, p > .001$).

\subsubsection{Mood}

If periphrastic \emph{doen} indeed has an interactional function, we would of course also expect it to occur more in interactional situations like imperatives and questions. In fact, it might even be the case that \emph{doen} is not so much an interaction marker, but is a marker for a less frequent mood - similar to how an accusative marker is a marker for a less frequent case. To investigate whether periphrastic \emph{doen} indeed is somehow related to the imperative or the interrogative mood, for each sentence the mood was annotated.

\begin{table}[h]
\begin{tabular}{|l||ll|ll|ll|}
\hline
&Indicative&&Imperative&&Interrogative&\\
\hline
periphrastic \emph{doen} sentences	&32 	& \emph{48.48\%}	&25		&\emph{37.88\%} & 9 &\emph{13.64\%}\\
control sentences					&91		& \emph{91\%}		&0	 	&\emph{0\%} 	& 9 &\emph{9\%}\\
\hline
\end{tabular}
\end{table}

Here we see indeed that non-indicative moods are much more frequent for \emph{doen} sentences than they are for the control sentences. A chi-square test showed this difference is significant ($\chi^2(2, N=166) = 48.37, p > .001$).

\subsubsection{Word order}

In the previous section we saw that the imperative and interrogative mood are more frequent for the \emph{doen} sentences than for the control sentences. Although this might be the result of periphrastic \emph{doen}'s interactional character, it might also hint at a more syntactic interpretation: in Dutch, both questions and imperatives are formed by deviating from the standard SVO word order found in main clauses: questions are formed with \emph{inversion}, which means that the inflected verb is in first position and the subject in second position (VSO), imperatives are formed by dropping the subject (VO).

\begin{exe}
\ex \gll Wil je een koekje?\\
Want you a cookie?\\
'Do you want a cookie?'

\ex \gll Eet je brood!\\
Eat your bread!\\
\end{exe}

Possibly, periphrastic \emph{doen} is used to indicate that a non-standard word order is used. To investigate this, all sentences where annotated for word order. Besides the reasons mentioned above, a frequent source of alternative word order turned out to be object fronting (OVS).

\begin{table}[h]
\begin{tabular}{|l||ll|ll|}
\hline
&SVO&&V(S)O or OVS&\\
\hline
periphrastic \emph{doen} sentences	&11		 &\emph{16.67\%}&55 	& \emph{83.33\%}	\\
control sentences					&58		 &\emph{58\%}	&42		& \emph{42\%}		\\
\hline
\end{tabular}
\end{table}

As can be seen from these data, whereas most \emph{doen} examples indeed have an alternative word order, this is not the case for the control sentences. A chi-square test showed this difference is significant ($\chi^2(1, N=166) = 27.97, p > .001$). It should be added, however, that this correlation does not tell us anything about causation. For example, it has been suggested above that the presence of emphatic \emph{doen} is caused by alternative word orders, but it might as well be the case the presence of \emph{doen} actually causes alternative word orders.

\subsection{Results} \label{regressionresults}

When investigated seperately, only the factor \emph{given before new} did not have a relation to the \emph{doen} sentences. When all factors are investigated at the same time in a multiple regression, however, we might see something different: some of the factors might be correlated, which means that adding them both does not add more predictive power to the model. Another advantage of multiple regression is that we can calculate $R^2$, which indicates how much of the variation the model can explain. This is particularly important to this study, as the significant results of the $\chi^2$ tests might have camouflaged that the results never were really good predictors: no category had a majority of the \emph{doen} sentences in group A, but a majority of the control sentences in group B.

\begin{table}[h] \centering
\begin{tabular}{|l||l|l|l|}
\hline
Factor&B&SE B& $\beta$ \\
\hline
Constant&0.03&0.07&\\
Phrasal verb&0.41&0.09&.29**\\
Given before new&0.03&0.07&.03\\
Habitual&0.21&0.07&.18*\\
Variety&0.22&0.07&.18*\\
Person&-0.27&0.09&-.20*\\
Mood&0.32&0.08&.29**\\
Inversion&0.06&0.07&.06\\
\hline
\end{tabular}
\caption{Multiple regression results, * $ p > 0.005$, **  $p > 0.001$ }
\end{table}
The most important thing to note here is that inversion is no longer significant. As discussed above, it is strongly related to person and mood, which may make it useless to a regression model; investigations of correlation data showed that this was indeed the case. As expected, this model has a relatively low predictive value: $R^2 = .461$.\\\indent

\section{The Memory Based Learning model} \label{mbl}

\subsection{Introduction}

Like regression models, the goal of Memory Based Learning models is to predict the value of a dependent variable (in our case, whether periphrastic \emph{doen} is used or not) for new examples. The way in which MBL tries to do this prediction is radically different, however. Regression models make predictions on the basis of abstractions. With every factor we include in the model, we assume another abstraction; for example, by including the factor person, we assume the existence of an abstract category person in language.\\\indent
Memory based learning doesn't do any of these assumptions. Instead of relying on human annotations for a dataset, it tries to make predictions on the basis of the data themselves. An MBL model thus is created by giving an algorithm as many examples as possible, and telling it to which class each example belongs. When given a new example, this example is compared to all examples seen in the training data. For the training examples that are most similar (that is, contain a lot of the same lexical items), it looks up the classes. The class that most of these training examples have is the class the MBL model predicts for the new example. An algorithm that predicts the class of new examples is called a \emph{classifier}. So to summarize, with this memory based learning model we are going to try to predict whether periphrastic \emph{doen} is used or not, only on the basis of lexical items.

\subsection{Method}

The classifier used is IB1-IG in the TiMBL software package \citep{d10}. To make the dataset ready for TiMBL, two manipulations were done to make sure TiMBL can only use the lexical items for its predictions (and no other clues):

\begin{itemize}

\item Of the control sentences, 34 sentences were randomly removed to create a balanced dataset. If we do not do this, 60.24\% of our data would be a control sentence, which would give the TiMBL an unfair advantage; by simply always predicting a control sentences, 60.24\% would be correct. This of course tells us nothing about the predictive power of the lexical items.
\item All \emph{doen} sentences were rewritten into a non-\emph{doen} sentence. This is necessary, because TiMBL can simply use the presence of the lexical item \emph{doen} as an indicator whether this is an periphrastic \emph{doen} sentence or not.

\end{itemize}

The resulting 132 sentences were split into a training set of 100 sentences and a test set of 32 sentences.

\subsection{Results}

TiMBL was able to correctly predict the sentence type for 24 of the 32 examples, which means an prediction accuracy of \textbf{75\%}.







%tabel labels?
%tabels too wide centeren

\section{Conclusion and discussion} \label{conc}
In this paper we saw that the periphrastic \emph{doen} construction, although impopular, can be found in (the eastern variant of) spoken Dutch. Of the various factors investigated, phrasal verbs, a possible habitual interpretation, a Dutch speaker, first or second person and non-indicative mood turned out to be useful predictors of the construction, but together could explain slightly less than half of the variation. On the other hand, a Memory Based Learning model using only lexical information, achieved much better results. This can either mean not the right factors have been investigated in the regression model, or that periphrastic \emph{doen} is triggered by certain lexical items.

\begin{thebibliography}{99}\footnotesize
\bibitem[CGN(2006)]{cgn}
\emph{Corpus Gesproken Nederlands, version 2.0.} Electronic resource. Available online at http://lands.let.ru.nl/cgn/home.htm.
\bibitem[Daelemans et al.(2010)]{d10}
Daelemans, W., J. Zavrel, K. Van der Sloot, \& A. Van den Bosch (2010). \emph{TiMBL: Tilburg Memory Based Learner, version 6.3, Reference Guide}. ILK Research Group Technical Report Series no. 10-01.
\bibitem[Duinhoven(1994)]{d94}
Duinhoven, A. M. (1994), Het hulpwerkwoord doen heeft afgedaan. \emph{Forum der Letteren 35}, 2: 110-131.
\bibitem[Cornips(1994)]{c94}
Cornips, L. (1994). De hardnekkige vooroordelen over de regionale doen+infinitief-constructie. \emph{Forum der Letteren. Jaargang 1994.}
\bibitem[Giesbers(1983)]{g83}
Giesbers, H. (1983-1984), Doe jij lief spelen? Notities over het perifrastisch doen. \emph{Mededelingen van de Nijmeegse Centrale voor Dialect- en Naamkunde 19}, 57-64.
\bibitem[ter Laan(1953)]{tl53}
Laan, K. ter (1953). \emph{Proeve van een Groninger spraakkunst}. Winschoten: van der Veen.
\bibitem[Nuijtens(1962)]{n62}
Nuijtens, E. (1962). \emph{De tweetalige mens}. Assen: Van Gorcum \& Comp.
\end{thebibliography}

\end{document}